\documentclass[11pt]{article}
\usepackage{setspace}
\usepackage{enumitem}
\usepackage{amsmath}
\usepackage{graphicx}
\usepackage{xcolor}
\setlength{\evensidemargin}{0 in}
\setlength{\oddsidemargin}{-0.1 in} \setlength{\textwidth}{6.7in}
\setlength{\textheight}{9.2in} \setlength{\topmargin}{-0.85in}

\begin{document}

\begin{center}	
\title{\textbf{What makes communities resilient to drought?}}

\author{Dan Blaustein-Rejito (GSPP), Ian Bolliger (ERG), Hal Gordon (ARE), \\Andy Hultgren (ARE), Yang Ju (Landscape Arch. and Env. Planning), \\Kate Pennington (ARE), Sara Stoudt (Statistics)}
\end{center}
\maketitle

\begin{abstract}

Drought has affected an unprecedented area of the United States over the past several years. In April 2016, 34 percent of the United States was abnormally dry and 14 percent was in drought.\footnote{US Drought Monitor 2016.}  This dryness impacts a large portion of the population: 84.3 million people live in drought-affected areas, and 17.5 million people live in areas experiencing "exceptional drought". \footnote{US Drought Monitor 2016.}   But the welfare impacts on the people who live in drought-affected communities are not purely determined by the severity and duration of the drought. What factors predict how severely a drought will impact a community?  This paper examines resilience to drought through a two-part analysis.  In the first stage, we find the correlation between drought realizations and changes key measures of welfare, including mortality, unemployment, homelessness, and water consumption.  Counties with low correlation can be thought of as drought-resilient, while high correlation indicates vulnerability.  In the second stage, we examine which characteristics predict resilience.  We discuss the predictive power of demographic characteristics such as race, income, age distribution, ....., for the impact of drought on welfare outcomes, with an eye to policy implications.  % WE WILL MODIFY THIS ONCE WE HAVE SOME RESULTS 2005-2014
\end{abstract}

\section{Why resilience matters}

Drought has affected an unprecedented area of the United States over the past several years. In April 2016, 14 percent of the United States was in drought and 34 percent was abnormally dry.\footnote{US Drought Monitor 2016.}  This dryness impacts a large portion of the population: 84.3 million people live in drought-affected areas, and 17.5 million live in areas experiencing "exceptional drought". \footnote{US Drought Monitor 2016.}   

% GENERATE AND INSERT A MAP HERE

The drought in California demands particular attention given its severity and its impact on national food production. California grows about half of the US's fruit, nuts, and vegetables and 22\% of dairy \footnote{EPA State Agricultural Profiles, https://www3.epa.gov/region9/ag/ag-state.html}. It has been in severe drought for the past five years.  Currently, 90\% of the state is in drought.  More than 50\% is in severe to exceptional drought. \footnote{US Drought Monitor 2016.} Some towns in the Central Valley can no longer supply running water to all their residents. Some communities have undertaken dramatic mandatory water rationing, and Governor Jerry Brown has requested--and achieved--domestic water consumption decreases of around 25\%.    

But the welfare impacts on the people who live in drought-affected communities are not purely determined by the severity and duration of the drought. Low-income communities, like East Porterville in the CA Central Valley, are more likely to lose tap water than wealthy neighborhoods of Los Angeles.  This study sets out to identify the factors that predict how severely a drought will impact a community.  A better understanding of risk and resilience can support more effective policy to prevent and counteract the welfare impacts of drought.

% AGAIN, WE'LL ADD ONCE WE HAVE A BASIC IDEA OF OUR FIRST STAGE RESULTS

\section{Data}
\begin{itemize} 
	\item Drought Monitor
	\\ The United States Drought Monitor combines a set of measures to categorize the severity of droughts in the US. Run in partnership between the National Drought Mitigation Center, the US Department of Agriculture, and the National Oceanic and Atmospheric Administration, a set of 250 experts reviews current data and revises the drought map every week. It was launched in 1999 to as an input for policymakers working on issues concerning water supply and drought.
	
	% YANG -- can you toss in a few bullet points here on the defintion of our drought variables?  I'll write them up into paragraphs
	\item NARR
	
	We use monthly temperature and precipitation from the NCEP North American Regional Reanalysis (NARR) project. 
	
	\item Yields
	
	We use annual, county-level crop yield data from the USDA for corn, soybeans, and wheat.
	
	\item Mortality
	
	We use county-level data from the CDC WONDER database for under-5, over-65, and all-ages mortality for 1999-2014. Metrics of mortality include total deaths, the crude mortality rate, and age adjusted mortality rates.

	\item ACS
	The American Community Survey is the annual survey conducted by the US Census. Data are available at the PUMA geographic level and are crosswalked to counties. Variables are weighted to represent the entire United States.  We use the ACS for demographic information such as population, age, race, and sex; employment and income information including employment in agriculture, annual household income, home ownership; and household water bills.

	\item Employment
	We use county-level data from the United States Bureau of Labor Statistics on the annual average total labor force, number employed, numer unemployed, and unemployment rate in each year of study.

	\item EPA
	
	We use geospatial data on facilities regulated by the United States Environmental Protection Agency to create a measure of the water intensiveness of the local economy.  We extract the FIPS code and the industrial category (SIC code) for each facility. Then we create a dummy variable for high water use and count the number of high-use facilities per county.

		The SICs that are coded high water- use are:
\begin{itemize}
\item Water-intensive industries:
\item 0100-0999 Agriculture, forestry, fishing
\item 2000-3999 Manufacturing
\item 4900-4932 Energy
	\end{itemize}
\end{itemize}	

There are several important issues with the use and interpretation of these data. We would like to clearly describe them up front. We decided to use the data despite these issues because it provided another opportunity to practice data management skills and because they may still provide some, albeit limited, insight.

First, these data are a snapshot of facilities managed in March, 2016. There is no way to gauge how the distribution of water-intensive industry has changed over time. This leads to two sample bias problems: first, the facilities contained in this data may not have existed during the window of our study. Second and most importantly, there may be an endogeneity problem where high water-use facilities in high-drought areas have shut down over the window of analysis and are missing from this sample. This would bias our estimates of the importance of water intensity down.

Finally, about 2/3 of the observations had to be dropped due to a missing FIPS or SIC code. If this censoring is nonrandom, it introduces an additional source of bias to the study.

\section{Model}

We develop a two-stage model.  In the first stage, we estimate county-level "vulnerability" to drought by regressing key metrics of well-being on drought measures.  In the second stage, we identify predictors of vulnerability.

\begin{itemize} \textbf{Stage 1: Estimating vulnerability}

In the first stage, we run separate specifications with three dependent variables: the mortality rate, employment rate, and crop yields.  We select mortality and employment because they can be argued to paint a picture of a county's general well-being.  We select crop yields because they should respond strongly to drought, and give us insight into the salience of drought for agricultural versus non-agricultural communities.

Our model is as follows:

\begin{equation}
y_{i,t} = \beta_{i} D_{i,t} + \alpha_i + \tau_i t + \gamma_{s,t} + \epsilon_{i,t} \label{Eq1}
\end{equation}
where $D_{i,t}$ refers to the number of days in U.S. Drought Survey bins 2-4 in county $i$ and year $t$, $\alpha_i$ are county fixed effects controlling for time-invariant differences between counties, $\tau_i$ is the coefficient on a county level linear time trend, and $\gamma_{s,t}$ are state-by-year fixed effects controlling for state level time trends common across all counties $i \in s$. Note that the state-by-year fixed effects will non-parametrically account for national trends in the outcome of interest as well as state-level trends. The identifying variation in this model is within-county annual deviation from the county time trend and from statewide annual average drought levels. %We correct standard errors for serial correlation over space and time, due to the spatial and temporal nature of droughts (neighboring observations in time and space are not independent draws).
\\

 The coefficients from these regresssions indicate how responsive our measures of well-being -- mortality, employment, and crop yields -- are to drought.  These measures of vulnerability will be used as the dependent variables in the second stage. 

\item \textbf{Second Stage: Identifying the predictors of vulnerability}

What accounts for this variation in vulnerability to drought?  We use the vulnerability estiamtes from the first stage as dependent variables in the second stage.  We select a vector of potential predictors and test their statistical and economic significance.  These predictors include county-level socioeconomic characteristics such as racial composition, age distribution, income, monthly water bill, and measures of home ownership, rural v. urban, and water-intensiveness of the local economy.

The model is as follows:

\begin{equation}
\beta_i = \rho_0 + \boldsymbol{\delta} \mathbf{X}_i + \nu_i \label{Eq2}
\end{equation}

where the $\beta_i$ are the coefficients from (\ref{Eq1}) for a given outcome; $\mathbf{X}_i$ is a vector of county-level demographic and economic information; and $\boldsymbol{\delta}$ denote the associated coefficients.\\
~\\
This regression is cross-sectional and therefore not well identified from a causal perspective. Any omitted variable that happens to covary with both the levels of $\beta_i$ and the variables $\mathbf{X}_i$ will bias the coefficients $\boldsymbol{\delta}$. However, this model will illustrate how "drought resilience" (a low value of $\beta_i$) covaries with a set of common county socioeconomic characteristics. The goal with this stage is to build insight regarding what characteristics are commonly associated with drought vulnerability. % Because county level characteristics are likely to be correlated over space, we anticipate correcting our OLS standard errors by clustering over space.
\end{itemize}


\section{Discussion}
\section{Conclusion}




University of Nebraska–Lincoln

quickstats.nass.usda.gov

\begin{thebibliography}{9}

\bibitem{ACS} 
Missouri Census Data Center. \textit{American Communities Survey}.
\\\texttt{http://mcdc.missouri.edu/websas/geocorr14.html}
 
\bibitem{mortality} 
Center for Disease Control (CDC), U.S. Department of Health and Human Services. 
\textit{Mortality for 1999 - 2014 with ICD 10 codes}. 
\\\texttt{ http://wonder.cdc.gov/mortSQL.html}

\bibitem{DroughtMonitor} 
National Drought Mitigation Center (NDMC), U.S. Department of Agriculture (USDA), and the National Oceanic and Atmospheric Administration (NOAA). \textit{U.S. Drought Monitor}.
\\\texttt{http://droughtmonitor.unl.edu/}

\bibitem{epa} 
Environmental Protection Agency (EPA). 
\textit{Facility Registry Service Spatial Data Download}. 
 \\\texttt{https://www.epa.gov/enviro/geospatial-data-download-service}
 
\end{thebibliography}

\end{document}
